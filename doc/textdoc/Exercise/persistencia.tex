
Muitas aplica��es utilizam a persist�ncia de dados para armazenar informa��es em um dispositivo de \textit{hardware} com a finalidade de recuperar essas informa��es em um momento determinado.

A linguagem de programa��o Java oferece diversas maneiras de programar a persist�ncia de objetos, dentre elas est�o a interface de programa��o de aplica��es (API) JDBC, o framework de persist�ncia Hibernate \cite{hibernate} e o framework de persist�ncia OJB \cite{ojb}.

Os exerc�cios apresentados neste documento utilizam a API JDBC. Essa API foi utilizada por estar dispon�vel na biblioteca padr�o da linguagem Java e possuir uma implementa��o mais simples em rela��o a abordagem de frameworks de persist�ncia.

A tecnologia JDBC fornece conectividade com os principais sistemas gerenciadores de banco de dados utilizados no mercado. Essa API permite que o desenvolvedor utilize tr�s recursos:

\begin{itemize}
	\item Estabelecer uma conex�o com um banco de dados.
	\item Enviar comandos SQL para o banco de dados.
	\item Processar os resultados.
\end{itemize}

Para maiores informa��es sobre a API JDBC utilize o link: \\
\indent \url{http://java.sun.com/products/jdbc/index.jsp}.